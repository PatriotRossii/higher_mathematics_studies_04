\documentclass{article}
\usepackage[utf8]{inputenc}

\usepackage[T2A]{fontenc}
\usepackage[utf8]{inputenc}
\usepackage[russian]{babel}

\usepackage{amsmath}

\usepackage{pgfplots}
\usepackage{sagetex}
\usepgfplotslibrary{fillbetween}
\usepgfplotslibrary{polar}
\pgfplotsset{compat=newest}

\usepackage{multienum}
\usepackage{geometry}
\geometry{
    left=1cm,right=1cm,top=2cm,bottom=2cm
}
\newcommand*\diff{\mathop{}\!\mathrm{d}}

\newtheorem{definition}{Определение}
\newtheorem{theorem}{Теорема}

\DeclareMathOperator{\sign}{sign}

\usepackage{hyperref}
\hypersetup{
    colorlinks, citecolor=black, filecolor=black, linkcolor=black, urlcolor=black
}

\title{Высшая математика}
\author{Лисид Лаконский}
\date{May 2023}

\begin{document}
\raggedright

\maketitle

\tableofcontents
\pagebreak

\section{Расчетно-графическая работа №4, вариант №21}

\subsection{Задание №1}

Дана аналитическая функция $f(z) = u + i v$, где $u = Re \ f(z)$, $v = Im \ f(z)$. Найти эту функцию, если:

$$
u = 4 x y - y
$$

\textbf{Решение.}

Найдем мнимую часть функции $f(z)$. Найдем частные производные от действительной части:

$\frac{\delta u}{\delta x} = 4 y$, $\frac{\delta u}{\delta y} = 4 x - 1$

Условия Коши-Римана:

$\frac{\delta v}{\delta y} = \frac{\delta u}{\delta x} = 4 y$

$\frac{\delta v}{\delta x} = -\frac{\delta u}{\delta y} = 1 - 4 x$

Поскольку $\frac{\delta v}{\delta y} = 4 y$, то общий интеграл $v(x, y)$ восстанавливаем частным интегрированием по «игрек»:

$v = \int (4 y) \diff y = 2 y^2 + \phi (x)$, где $\phi (x)$ — неизвестная функция, зависящая только от «икс».

Теперь от $v(x, y) = 2 y^2 + \phi (x)$  берем частную производную по «икс»:

$\frac{\delta v}{\delta x} = \phi'_x(x)$

И результат приравниваем к изначальной частной производной по «икс»:

$\phi'_x(x) = 1 - 4 x \Longleftrightarrow \phi(x) = x - 2x^2 + C$

$v(x, y) = 2 y^2 + x - 2x^2 + C$

Таким образом, $f(z) = u (x, y) + v (x, y) i = (4 x y - y) + (2 y^2 + x - 2x^2 + C) i = 4 x y - y + 2y^2 i + x i - 2x^2 i + C i = y (4x - 1) + 2 i (y^2 - x^2) + x i + C i$

\subsection{Задание №2}

Найти образ $E$ области $D$ плоскости $z$ при отображении функцией $w = f(z)$ 

$$
D \ : \ \{ z | \ |z| < 1; \ 0 < arg \ z < \frac{\pi}{4} \}; \ w = z^4
$$

\textbf{Решение.}

$w = z^4 = |z|^4 (\cos 4 \arg z + i \sin 4 \arg z)$, то есть, $0 < |w| < 1$, $0 < Arg \ w < \pi$.

Множество $E$ есть единичный ($r = 1$) полукруг.

\subsection{Задание №3}

Найти функцию, отображающую область $D$ плоскости $z$ на область $E$ плоскости $w$.

$$
D \ : \ \{ z \ | \ | z - 2 i | < 1, \ Re \ z < 0 \}; \ \ E: \{ w \ | \ Im \ w > 0 \}
$$

\end{document}